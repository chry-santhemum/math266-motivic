\section{Hamilton Wan: Unimodular Rows and Motivic Homotopy Theory}

\subsection{Motivation and Background}

\subsubsection{Why Unimodular Rows?} 

To motivate the study of unimodular rows, we first review a foundational question in algebraic K-theory that we encountered at the start of this course. 

\begin{question}[Serre's Problem]
    Let $k$ be a field. Is every finitely generated projective $k[t_1,\ldots,t_n]$-module free? In other words, is the base change functor \[\operatorname{Mod}_k^{\text{f.g., proj}} \to \operatorname{Mod}_{k[t_1,\ldots,t_n]}^{\text{f.g., proj}}, \quad M \mapsto M \otimes_k k[t_1,\ldots,t_n]\] essentially surjective?
\end{question}

Note that this problem can be reinterpreted geometrically as asking whether every algebraic vector bundle on $\A^n_k$ is trivial. On the other hand, from a motivic perspective, Serre's problem asks about $\A^1$-invariance of the sheaf $H^1_{\mathrm{Nis}}(-,\mathrm{GL}_*)$ taking an affine scheme $\Spec(R)$ to the graded abelian group of its algebraic vector bundles. When $n = 1$, this problem is straightforward, as $k[t_1]$ is a PID. For arbitrary $n$, our intuition from topology tells us that vector bundles on $\A^n_k$ should all be trivial because $\A^n_k$ is somehow a ``contractible" space. Of course, this intuition is far from a rigorous proof, and the first solutions to Serre's problem required deeper mathematics. In 1976, Quillen \cite{quillen_serre_problem} and Suslin \cite{suslin_serre_conj} produced independent proofs of the following theorem, answering Serre's question in the positive.

\begin{theorem}[Quillen--Suslin]
    Every finitely generated projective $k[t_1,\ldots,t_n]$-module is free.
\end{theorem}

Quillen's proof of this theorem was quite sophisticated, introducing an ingenious technique called Quillen patching. Following Suslin and Quillen's original proofs in 1976, Suslin and Vaserstein independently produced new, elementary proofs of the Quillen--Suslin theorem. Roughly speaking, Suslin (in a letter to Bass dated May 2, 1976, apparently\footnote{at least, this is how the proof was cited in \cite[Chapter III]{Lam_Serre_Problem}}) provided a ``linear-algebraic" proof showing that the ring $k[t_1,\ldots,t_n]$ is a \textit{Hermite ring}, which means that unimodular rows valued in the ring can be completed to invertible matrices; it is known that finitely generated projective modules over a Hermite ring are free. Soon after, Vaserstein\footnote{in unpublished notes cited by \cite[Chapter III]{Lam_Serre_Problem}} provided another elementary proof (colloquially known as ``Vaserstein’s 8-line proof") of the Quillen--Suslin theorem using the theory of unimodular rows. The details of Suslin's and Vaserstein's proofs are beyond the scope of this appendix (in particular, not germane to motivic homotopy theory), and we refer the reader to \cite[Chapter III]{Lam_Serre_Problem} for an excellent exposition of these proofs. However, we use their proofs as motivation for studying unimodular rows. At a very high level, unimodular rows in a $k$-algebra $A$ control the surjective ring homomorphisms $A^{\oplus n} \to A$ and are thus ubiquitous in algebraic $K$-theory. We will discuss toward the end of this note some applications to the study of stably free modules over a smooth algebra, for instance.

\subsubsection{Background} 

Let $A$ be a commutative algebra over a field $k$, and choose a positive integer $n$. 

\begin{definition}
    A \textit{unimodular row} of length $n$ valued in $A$ is a sequence $(a_1,\ldots,a_n)$ of elements of $A$ that collectively generate the unit ideal in $A$.
\end{definition}

 Observe that unimodular rows classify surjective $A$-module homomorphisms $A^n \to A$: the unimodular row $(a_1,\ldots,a_n)$ corresponds to the $A$-module morphism sending the $i$th standard basis vector to $a_i$. From a geometric perspective, we can think of unimodular rows as corresponding to the morphisms $\Spec(A) \to \A^n_k$ that factor (uniquely) through the Zariski open subset $\A^n_k\setminus\{0\}$. Indeed, if we identify $\A^n_k$ with the variety of $1 \times n$ row matrices, then $\A^n_k\setminus\{0\}$ consists precisely of the full rank matrices, i.e., the ones parameterizing surjective homomorphisms $A^n \to A$. Thus, the following proposition is manifest.

 \begin{proposition}\label{prop:unimodular-basic}
     The scheme $\A^n_k\setminus\{0\}$ represents the  functor $\operatorname{AffSm}^{\op}_k \to \mathsf{Set}$ given by $\Spec A \mapsto \operatorname{Um}_{n}(A)$. 
 \end{proposition}

In fact, even more is true. Observe that the set of unimodular rows, which we denote by $\mathrm{Um}_n(A)$, carries a natural action of $\mathrm{GL}_n(A)$ on the right. We consider the subgroup $E_n(A) \subset \mathrm{GL}_n(A)$ generated by the \textit{elementary matrices}, that is, those matrices implementing elementary row operations. The main object of interest in this appendix is the set of orbits $\mathrm{Um}_n(A)/E_n(A)$. Since unimodular rows belonging to the same orbit give rise to the ``same" surjective homomorphism $A^n \to A$, we may instead be interested in understanding the functor $\Spec A \mapsto \mathrm{Um}_n(A)/E_n(A)$. 

\begin{question}\label{question:q1}
    Is there an equivalence relation $\sim$ on $\Hom(\Spec A,\A^n\setminus\{0\})$ so that $\Hom(\Spec A,\A^n\setminus\{0\})/\sim$ is naturally in bijection with $\mathrm{Um}_n(A)/E_n(A)$?
\end{question}

It turns out that this question can be answered in the language of motivic homotopy theory. Moreover, if $n$ is sufficiently large, $\mathrm{Um}_n(A)/E_n(A)$ can be endowed with the structure of a group using elementary but slightly convoluted means. Motivic homotopy theory sheds light on this group structure, giving it a concrete geometric interpretation. In this appendix, we explore these motivic perspectives on unimodular rows. 

\subsubsection{Some Topological Motivation}

To motivate the results we discuss in this appendix, we make a quick topological digression. For a finite CW complex $X$, let $C(X)$ denote the ring of continuous functions $X \to \R$. Under the assumption $\dim(X) \leq 2n - 4$, van der Kallen shows that the set $\mathrm{Um}_n(C(X))/E_n(C(X))$ carries the natural structure of an (abelian) group. On the other hand, recall that the topological $m$-sphere $S^m$ carries the natural structure of a co-H-space thanks to the natural fold map $S^m \vee S^m \to S^m$. Under suitable conditions on the space $X$, the structure above endows the set of homotopy classes of pointed maps $X \to S^m$, denoted $\pi^m(X)$, with the structure of a group. For instance, a classical theorem of Borsuk \cite{borsuk} shows that $\dim X \leq 2n - 4$ is sufficient for $\pi^n(X)$ to carry the aforementioned group structure. In this topological setting, van der Kallen \cite[Theorem 7.7]{vdk} proves

\begin{theorem}[van der Kallen, 1989]\label{thm:vdk}
    Let $X$ be a $d$-dimensional CW complex. Then, there exists a natural bijection \[
\mathrm{Um}_n(C(X))/E_n(C(X)) \cong \pi^{n-1}(X),
\]
which can be upgraded to a group isomorphism whenever $d \leq 2n-4$.
\end{theorem}

Van der Kallen asked whether his theorem above can be extended to the algebraic setting. At the time, there was no suitable scheme-theoretic notion of cohomotopy groups, or even spheres. However, once the tools of $\A^1$-invariant motivic homotopy theory were developed, van der Kallen's question found a resolution in motivic homotopy theory, owing to Fasel \cite{Fasel_2010} and Lerbet \cite{LERBET2024109415}. Let's set up the algebraic analogue of van der Kallen's result. Clearly, we want to replace $X$ with the spectrum of a commutative $k$-algebra $A$. On the other hand, the stable dimension of $A$ ought to play the role of the dimension of the CW complex $X$. The most subtle step is determining the appropriate replacement for the topological $n$-sphere $S^n$. It turns out that the correct choice is the motivic sphere $S^{2n-1,n} = \A^n \setminus \{0\}$. The most basic result that we study in this note is the proof of the following theorem of Fasel, which in some sense, provides an aswer to a motivic version of Question \ref{question:q1}.

\begin{theorem}[Fasel, 2010]\label{thm:fasel}
    There exists a natural bijection
    \begin{align}\label{eqn:fasel_bijection}\mathrm{Um}_n(A)/E_n(A) \xto{\sim} [\Spec A, \A^n_k\setminus \{0\}]_{\A^1},\end{align} 
    where the right-hand side denotes the set of $\A^1$-homotopy classes of maps $\Spec A \to \A^n_k \setminus\{0\}$.
\end{theorem}

In this sense, we can say that $\A^n_k \setminus\{0\}$ represents the functor $k\mathsf{-Alg} \to \mathsf{Set}$ given by $A \mapsto \mathrm{Um}_n(A)/E_n(A)$ in the category of motivic spaces. Under some dimensionality assumptions on the ring $A$, van der Kallen endows the left-hand side of (\ref{eqn:fasel_bijection}) with the structure of an abelian group. With the same assumptions, the right-hand side can also be equipped with the structure of an abelian group coming from the co-H-space structure on $\A_k^n\setminus\{0\}$. Completing the analogy with van der Kallen's topological result, Lerbet \cite{LERBET2024109415} shows that Fasel's bijection is in fact a group isomorphism whenever these group structures can be defined. That is, our goal is to build up to the following theorem.

\begin{theorem}[Lerbet, 2024]
    Suppose $A$ has Krull dimension at most $2n-4$. Then, the bijection (\ref{eqn:fasel_bijection}) is a group isomorphism.
\end{theorem}

Since the group structure on the right-hand side of (\ref{eqn:fasel_bijection}) can be interpreted geometrically, we can think of Lerbet's theorem as a motivic re-interpretation of van der Kallen's group law. 

A natural question is to ask for an explicit computation of the group $\operatorname{Um}_n(A)/E_n(A)$. In the special case $n = \dim(A) + 1$, Fasel \cite{Fasel_2010} produces a cohomological description of this group.

\begin{theorem}[Fasel, 2010]\label{thm:fasel_cohomology}
    Let $k$ be a perfect field of characteristic not equal to two, and suppose $A$ is a smooth $k$-algebra of Krull dimension $d = n-1 \geq 2$. There exists a natural isomorphism 
    \[
    \mathrm{Um}_n(A)/E_n(A) \xto{\sim} H^{n-1}(\Spec A,\KMW_n),
    \]
    where $\KMW_\ast$ is the Milnor--Witt sheaf.
\end{theorem}

Using this theorem, Fasel explicitly computes the group $\mathrm{Um}_n(A)/E_n(A)$ in some exceptional cases. Fasel's work in this direction, building on previous work of Morel, also has some applications to the study of stably free modules, which we briefly mention at the end of this note.

\subsubsection{Outline} Before we proceed, we provide a brief overview of the structure of this note. In Subsection \ref{sec:fasel}, we will first prove Fasel's bijection (\ref{eqn:fasel_bijection}). In Subsection \ref{sec:group_structure}, we discuss the group structure on $\mathrm{Um}_n(A)/E_n(A)$ discovered by van der Kallen. In Subsection \ref{sec:lerbet}, we provide a brief exposition of Lerbet's proof that Fasel's bijection is actually a group isomorphism whenever the object involved carry a group structure. Finally, in Subsection \ref{sec:cohomology}, we discuss Fasel's cohomological interpretation of unimodular rows, some explicit calculations of the group structure, and applications to stably free modules.  

\subsection{Unimodular Rows and \texorpdfstring{$\A^1$}{A1}-Homotopy Classes}\label{sec:fasel}

In this relatively brief section, we provide an exposition of Fasel's proof of Theorem \ref{thm:fasel}, which establishes the following bijection for any commutative $k$-algebra $A$:
\[
\mathrm{Um}_n(A)/E_n(A) \xto{\sim} [\Spec A, \A^n_k\setminus \{0\}]_{\A^1},
\]
Note that Theorem \ref{thm:fasel} holds without any restrictions on the stable dimension of $A$.

Strictly speaking, Fasel establishes a more concrete version of the bijection above, in terms of naive homotopy classes of maps $\Spec A \to \A^n_k\setminus\{0\}$. As we will soon see, his result is readily reframed in terms of honest $\A^1$-homotopy classes, and we choose to reframe his result in these terms since (1) it keeps in line with the content of this course and (2) more importantly, it sets up the framework for Lerbet's group structure theorem. Recall that a naive $\A^1$-homotopy between maps $f,g\colon X \to Y$ of $k$-schemes is a morphism $H\colon X \times_k \A^1_k \to Y$ such that the following diagram commutes:
% https://q.uiver.app/#q=WzAsNCxbMywxLCJZIl0sWzAsMCwiWCBcXHRpbWVzXFx7MFxcfSJdLFswLDIsIlggXFx0aW1lcyBcXHsxXFx9Il0sWzEsMSwiWCBcXHRpbWVzXFxtYXRoYmJ7QX1eMV9rIl0sWzEsMywiIiwwLHsic3R5bGUiOnsidGFpbCI6eyJuYW1lIjoiaG9vayIsInNpZGUiOiJib3R0b20ifX19XSxbMSwwLCJmIFxcdGltZXMgXFxvcGVyYXRvcm5hbWV7aWR9X3tcXHswXFx9fSJdLFsyLDAsImcgXFx0aW1lcyBcXG9wZXJhdG9ybmFtZXtpZH1fe1xcezFcXH19IiwyXSxbMywwLCJIIiwxXSxbMiwzLCIiLDIseyJzdHlsZSI6eyJ0YWlsIjp7Im5hbWUiOiJob29rIiwic2lkZSI6InRvcCJ9fX1dXQ==
\[\begin{tikzcd}
	{X \times\{0\}} \\
	& {X \times\mathbb{A}^1_k} && Y \\
	{X \times \{1\}}
	\arrow[hook', from=1-1, to=2-2]
	\arrow["{f \times \operatorname{id}_{\{0\}}}", from=1-1, to=2-4]
	\arrow["H"{description}, from=2-2, to=2-4]
	\arrow[hook, from=3-1, to=2-2]
	\arrow["{g \times \operatorname{id}_{\{1\}}}"', from=3-1, to=2-4]
\end{tikzcd}\]
In ring-theoretic terms, if $Y = \Spec B$ and $X = \Spec A$, a naive $\A^1$-homotopy between the corresponding ring maps $f^\sharp, g^\sharp\colon B \to A$ amounts to a map $H^\sharp\colon B \to A[t]$ such that $H^\sharp(0) = f^\sharp$ and $H^\sharp(1) = g^\sharp$. Here (and henceforth), for any $\alpha \in A$, we use the notation $H^\sharp(\alpha)\colon B \to A$ to denote the composition of $H^\sharp$ with the quotient map $A[t] \to A[t]/(t - \alpha) \cong A$.

\begin{proposition}\cite[Theorem 2.1]{Fasel_2010}\label{prop:fasel_naive}
    There exists natural bijection
    \[
    \mathrm{Um}_n(A)/E_n(A) \xto{\sim} [\Spec A, \A^n_k\setminus \{0\}]_{N},
    \]
    where the right-hand side denotes the set of \textit{naive} $\A^1$-homotopy classes of maps $\Spec A \to \A^n_k \setminus\{0\}$. 
\end{proposition}

\begin{remark}
    In fact, Fasel proves this proposition in greater generality. Let $\mathrm{Um}_{m,n}(A)$ denote the set of surjective $A$-module morphisms $A^n \to A^m$ and let $D(m,n) = \A^{m \times n}\setminus V(m,n)$, where $V(m,n)$ is the vanishing locus of all $m \times m$ minors in $\A^{m \times n} = \operatorname{Mat}_{m \times n}$. In \cite[Theorem 2.1]{Fasel_2010}, he establishes a natural bijection
    \[
    \operatorname{Um}_{m,n}(A)/E_n(A) \xto{\sim} [\Spec A, D(m,n)]_{N}.
    \]
    Note that the case $m = 1$ is our Proposition \ref{prop:fasel_naive}. 
\end{remark}

\begin{proof}[Proof of Proposition \ref{prop:fasel_naive}]
    We follow the proof given by Fasel in \cite[Theorem 2.1]{Fasel_2010}, adding more detail where appropriate. Thanks to Proposition \ref{prop:unimodular-basic}, it suffices to prove that (1) unimodular rows belonging to the same orbit define naively homotopic morphisms and (2) if two unimodular rows define naively homotopic morphisms, then they belong to the same orbit. Given a unimodular row $u \in \operatorname{Um}_n(A)$, let $\psi_u\colon \Spec A \to \A^n\setminus\{0\}$ denote the induced morphism. 

    Claim (1) is relatively straightforward. Letting $E_n$ denote the group scheme of elementary matrices, we can understand elements of $E_n(A)$ as morphisms $M\colon \Spec A \to E_n$. Given a unimodular row $u \in \mathrm{Um}_n(A)$, the morphism $\psi_{uM}\colon \Spec A \to \A^n\setminus\{0\}$ is the composition
    \[
    \Spec A \xto{\Delta} \Spec A \times \Spec A \xto{\psi_u \times M} \A^n\setminus\{0\} \times E_n \to \A^n \setminus \{0\}, 
    \]
    where $\Delta$ is the diagonal morphism and the last morphism the natural right action of $E_n$ on $\A^n \setminus\{0\}$. An explicit construction shows that any elementary is naively homotopic to the identity. Thus, the composition above is naively homotopic to $\psi_u$, as desired.
    
    Claim (2) is a bit trickier. Observe that a naive homotopy between unimodular rows $u_0$ and $u_1$ amounts to the data of a unimodular row $U(t) \in A[t]$ such that $U(0) = u_0$ and $U(1) = u_1$. Consider the corresponding (split) short exact sequence of $A[t]$-module maps
    \[
    0 \to B \to A[t]^n \xto{U(t)} A[t] \to 0,
    \]
    where $B$ is the kernel of $U(t)$. Since the short exact sequence above splits, we see that the submodule $B \subset A[t]$ is a finitely-generated projective $A[t]$-module. A result of Lindel \cite[Lemma 3]{Lindel} shows that $B$ is extended from $A$ in the sense that $B \cong B(0)[t]$, where $B(0)$ denotes the quotient $B/tB$. Thus, we have a split short exact sequence
    \[
    0 \to B(0) \to A^n \xto{U(0) = u_0} A \to 0.
    \]
    Tensor this sequence up by $A[t]$. Observe that the \textit{same} copy of $B(0)[t] = B \subset A$ is the kernel of both $U(t)\colon A[t]^n \to A[t]$ and the constant map $u_0\colon A[t]^n \to A[t]$. Since both maps induce a splitting $A[t]^n \cong A[t] \oplus B$, there must exist an automorphism $\alpha\colon A[t]^n \to A[t]^n$ fixing $B$ and satisfying $u_0 \circ \alpha = U(t)$. Moreover, note that $\psi_0$ is the identity since $U(0) = u_0$. A result of Vorst \cite{Vorst} shows that $\alpha$ can be identified with an elementary matrix $E(t) \in E_n(A[t])$. It follows that $u_0 \circ E(1) = U(1) = u_1$, as desired.
\end{proof}

To re-express Fasel's result about naive homotopy classes as a result about $\A^1$-homotopy classes, we will use the fact that $\A^n \setminus \{0\}$ is $\A^1$-naive. Recall by \autoref{def:a1-naive} that a simplicial presheaf $F \in \PSh(\Sm_k)$ is called \textit{$\A^1$-naive} if for every smooth affine $k$-scheme $U$, we have an equivalence
\[
L_{\A^1}(F)(U) \xto{\sim} L_{\operatorname{mot}}(F)(U).
\]
For our purposes, the most important consequence is that naive homotopy classes of maps to $\A^1$-naive motivic spaces are in bijection with $\A^1$-homotopy classes of maps: that is, there exists a natural bijection
\[
[\Spec A, \A^n_k\setminus \{0\}]_{N} \xto{\sim} [\Spec A, \A^n_k\setminus \{0\}]_{\A^1}.
\]

\begin{lemma}
    Let $k$ be a field. The scheme $\A^n_k \setminus \{0\}$ is $\A^1$-naive.
\end{lemma}

\begin{proof}
    This is a direct application of \cite[Corollary 4.2.6]{AHW2}.
\end{proof}

Combining the lemma above with Proposition \ref{prop:fasel_naive}, we have proven

\begin{corollary}[Theorem \ref{thm:fasel}]
    There exists a natural bijection \[\mathrm{Um}_n(A)/E_n(A) \xto{\sim} [\Spec A, \A^n_k\setminus \{0\}]_{\A^1}.\] 
\end{corollary}

\subsection{Van Der Kallen's Group Structure}\label{sec:group_structure}

In this section, we discuss the natural group structure on $\mathrm{Um}_n(A)/E_n(A)$ discovered by van der Kallen \cite{vdk}. As we mentioned in the introduction, van der Kallen discovered this group structure while studying the orbits of unimodular rows for the ring of continuous functions on a finite CW complex. The exposition in this section closely follows \cite[Section 3]{LERBET2024109415}. Van der Kallen's group structure was used by Lerbet to fully generalize Theorem \ref{thm:vdk} to the algebraic setting. The key idea here is to construct a natural correspondence between $\mathrm{Um}_n(A)/E_n(A)$ and a certain universal group of maps from $\mathrm{Um}_n(A)$. Although the structure of this universal group seems quite daunting \textit{a priori}, we will introduce a convenient trick (the Mennicke--Newman lemma) for computing products of orbits of unimodular rows at the end of this section.

\subsubsection{Weak Mennicke Symbols}

\begin{definition}\label{def:wms}
    Let $G$ be a group. A \textit{weak Mennicke symbol} on $\mathrm{Um}_n(A)$ is a map of sets $\mu\colon \mathrm{Um}_n(A) \to G$ satisfying the following relations:
    \begin{itemize}
        \item[(i)] The map $\mu$ is invariant under the right-action of $E_n(A)$. That is, for any elementary matrix $E$ and for any $u \in \mathrm{Um}_n(A)$, we have $\mu(uE) = \mu(u)$. In other words, the map $\mu$ factors through the quotient $\mathrm{Um}_n(A) \to \mathrm{Um}_n(A)/E_n(A)$. 
        \item[(ii)] For any pair of unimodular rows $u,u' \in \mathrm{Um}_n(A)$ of the form $u = (a,u_2,\ldots,u_n)$ and $u' = (1+a,u_2,\ldots,u_n)$ and for any $r \in A$ such that $r(1 + a) = a$ modulo the ideal $\langle u_2,\ldots,u_n\rangle$, we have $\phi(u) = \phi(u'')\phi(u')$, where $u'' = (r,u_2,\ldots,u_n)$ (it can be shown that this sequence is also a unimodular row). 
    \end{itemize}
\end{definition}

\begin{remark}
    Van der Kallen uses the terminology ``weak" Mennicke symbol because the axioms above do indeed give a weaker definition of the Mennicke symbols introduced by Suslin \cite{suslin2006mennicke} for the algebraic $K$-theory of fields.
\end{remark}

A basic, yet fundamental, insight is the existence of a universal group $\mathrm{WMS}_n(A)$ of weak Mennicke symbols.

\begin{definition}
    For any positive integer $n$ and any $k$-algebra $A$, define the group $\mathrm{WMS}_n(A)$ as the quotient of the free group on $\mathrm{Um}_n(A)$ by the relations generated by Conditions (i) and (ii) in Definition \ref{def:wms}. Denote the class of the element corresponding to $u\in \mathrm{Um}_n(A)$ by $[u]$.
\end{definition}

There exists a universal weak Mennicke symbol $\hat{\mu}\colon \mathrm{Um}_n(A) \to \mathrm{WMS}_n(A)$ given by $u \mapsto [u]$. This symbol is universal in the sense that for any weak Mennicke symbol $\mu\colon \mathrm{Um}_n(A) \to G$, there exists a unique group homomorphism $\mu'\colon \mathrm{WMS}_n(A) \to G$ such that $\mu' \circ \hat{\mu} = \mu$. The key insight of van der Kallen is that the group structure on $\mathrm{WMS}_n(A)$ can be transferred to the set $\mathrm{Um}_n(A)/E_n(A)$ under some assumptions on $A$. To properly state these assumptions, we need to define the \textit{stable dimension} of $A$.

\begin{definition}
    Let $A$ be a commutative $k$-algebra. The stable rank of $A$, denoted $\operatorname{srank}(A)$, is defined as the smallest integer $r$ such that for any unimodular row $(a_1,\ldots,a_{r+1})$, there exist $b_1,\ldots,b_r$ such that $(a_1 + b_1a_{r+1}, a_1 + b_2a_{r+1},\ldots, a_{r}+b_{r+1}a_{r+1})$ is also unimodular. If no such integer $r$ exists, then we declare $\operatorname{srank}(A) = \infty$. The stable dimension of $A$, denoted $\operatorname{sdim}(A)$, is defined as $\operatorname{srank}(A) - 1$. 
\end{definition}

\begin{remark}
    For our purposes, we simply need an upper bound on the stable dimension of $A$. Bass \cite[Theorem 1]{bass} shows that the stable dimension of a Noetherian ring is bounded above by its Krull dimension, and thus, in applications, we can (and will) use the Krull dimension of $A$ instead. We record the definition of the stable dimension for completeness.
\end{remark}

When the stable dimension of $A$ is bounded, van der Kallen \cite[Theorems 3.6 and 4.1]{vdk} equips $\mathrm{Um}_n(A)/E_n(A)$ with the structure of an abelian group. His results are summarized in the following theorem. The proof, while completely ``elementary," is quite long and computational, so we omit it from our exposition and refer the interested reader to \cite[Sections 3 and 4]{vdk}.

\begin{theorem}[Van der Kallen, 1989]
    Suppose $\operatorname{sdim}(A) \leq 2n-4$. Then, the group $\operatorname{WMS}_n(A)$ is abelian, and the universal map $\operatorname{Um}_n(A)/E_n(A) \to \operatorname{WMS}_n(A)$ is a bijection. In particular, $\operatorname{Um}_n(A)/E_n(A)$ inherits the structure of an abelian group.
\end{theorem}

\subsubsection{The Mennicke--Newmann Lemma: Expliciting the Group Law}

Despite the abstract definition of $\operatorname{WMS}_n(A)$, it turns out that the group law on $\operatorname{Um}_n(A)/E_n(A)$ can be computed rather explicitly using two facts.

First, \cite[Lemma 3.5(v)]{vdk} and \cite[Lemma 3.1]{vdk2} imply that the following relation holds in $\operatorname{WMS}_n(A)$:
\begin{gather}\label{eqn:mn_relation}
\text{For any pair of unimodular rows of the form $u = (a,u_2,\ldots,u_n)$ and $u' = (1-a,u_2,\ldots,u_n)$, we have}\\ 
\label{eqn:mn_relation2}\begin{align}
    [u][u'] = [(a(1-a),u_2,\ldots,u_n)].
\end{align}
\end{gather}

Second, we have the so-called Mennicke--Newman lemma. The proof of this lemma will give us a good handle on elementary calculations involving unimodular rows, and thus, we provide it for pedagogical reasons.

\begin{lemma}[Mennicke--Newman]\label{lem:mn}
    Assume that $A$ has stable dimension $d \leq 2n-3$, and let $U,U' \in \mathrm{Um}_n(A)$ be unimodular rows. Then, there exist elementary matrices $E,E' \in E_n(A)$ such that $UE = (a,r_2,\ldots,r_n)$ and $U'E' = (1-a,r_2,\ldots,r_n)$ for some $a,r_2,\ldots,r_n \in A$.
\end{lemma}

This version of the Mennicke--Newman lemma and its proof appears as Proposition 3.6 in \cite{LERBET2024109415}.

\begin{proof}
    As mentioned above, we follow the proof of \cite[Proposition 3.6]{LERBET2024109415}. The proof is accomplished in three steps. Fix unimodular rows $U = (u_1,\ldots,u_n)$ and $U' = (v_1,\ldots,v_n)$.

    \textbf{Step 1.} We claim that it suffices to assume that $(u_2,\ldots,u_n,v_2,\ldots,v_n)$ is a unimodular row of length $2n-2$. Observe first that $(u_2,\ldots,u_n,v_2,\ldots,v_n, u_1v_1)$ is certainly a unimodular row -- indeed, if $b_1,\ldots,b_n$ and $c_1,\ldots,c_n$ are such that $\sum_i b_iu_i = 1$ and $\sum_i c_iv_i = 1$, then \[\sum_{j=2}^n \sum_{i=2}^n (b_iu_ic_j)v_j + b_1c_1(u_1v_1)= \sum_{j=1}^n c_j\left(\sum_{i=1}^n b_iu_i\right)v_j =1. \]
    Since $A$ has stable dimension at most $2n-3$, it has stable rank at most $2n-2$. In particular, by definition, there exist $\alpha_2,\ldots,\alpha_n,\beta_2,\ldots,\beta_n \in A$ such that
    \[
    (u_2+\alpha_2u_1v_1, \ldots,u_n+\alpha_nu_1v_1,v_2+\beta_2u_1v_1,\ldots,v_n + \beta_nu_1v_1)
    \]
    is a unimodular row. Note that $(u_1,u_2+\alpha_2u_1v_1, \ldots,u_n+\alpha_nu_1v_1)$ belongs to the $E_n(A)$-orbit of $(u_1,\ldots,u_n)$, so we may as well replace $(u_1,\ldots,u_n)$ by $(u_1,u_2+\alpha_2u_1v_1, \ldots,u_n+\alpha_nu_1v_1)$. Similarly, we may replace $(v_1,\ldots,v_n)$ by $(v_1,v_2+\beta_2u_1v_1,\ldots,v_n + \beta_nu_1v_1)$. Ultimately, we are only concerned with the $E_n(A)$-orbits of $U$ and $U'$, so these substitutions complete the first step. 

    \textbf{Step 2.} Assuming Step 1, we claim that it suffices to assume $u_1 + v_1 = 1$. The unimodularity of the row $(u_2,\ldots,u_n,v_2,\ldots,v_n)$ implies that there exist $\gamma_i,\delta_i \in A$ such that
    \[
    \sum_{i=2}^n (\gamma_iu_i + \delta_iv_i) = u_1 + v_1 - 1.
    \]
    Observe that $\left(u_1-\sum_{i=1}^n\gamma_iu_i, u_2,\ldots,u_n\right)$ and $\left(v_1-\sum_{i=1}^n\beta_iv_i, v_2,\ldots,v_n\right)$ belong to the $E_n(A)$-orbits of $(u_1,\ldots,u_n)$ and $(v_1,\ldots,v_n)$, respectively, and that they satisfy the conditions of Steps 1 and 2. Thus, we may replace $(u_1,\ldots,u_n)$ by $\left(u_1-\sum_{i=1}^n\gamma_iu_i, u_2,\ldots,u_n\right)$ and replace $(v_1,\ldots,v_n)$ by $\left(v_1-\sum_{i=1}^n\beta_iv_i, v_2,\ldots,v_n\right)$ to complete the second step. 

    \textbf{Step 3.} Finally, we establish the claim of the Mennicke--Newman Lemma under the simplifying assumptions of Steps 1 and 2. Set $a := u_1$. Since $u_1 + v_1 = 1$, we have
    \[
    u_i + v_i - (u_i + v_i)(u_1 + v_1) = 0
    \]
    for each $i = 2,\ldots,n$. In particular, we can define $r_i \in A$ to be the element 
    \[
    u_i - u_1(u_i-v_i) = v_i + v_1(u_i-v_i). 
    \]
    Then, \[(a,r_2,\ldots,r_n) = (u_1,u_2-u_1(u_2-v_2), \ldots,u_n-u_1(u_n-v_n))\] is obtained from $(u_1,\ldots,u_n)$ by the action of the elementary lower triangular matrix with ones on the diagonal, $v_i-u_i$ in the $i$th entry of the first column for $i \geq 2$, and zeros elsewhere. Similarly, \[(1-a,r_2,\ldots,r_n) = (v_1,v_2 + v_1(u_i-v_i), \ldots,v_n + v_1(u_i-v_i)\] belongs to the $E_n(A)$-orbit of $(v_1,\ldots,v_n)$. This completes the proof.
\end{proof}

\begin{remark}\label{rem:no_dim_assumption}
    Observe that Step 1 was the only step that used the fact that $\mathrm{sdim}(A) \leq 2n-3$. In particular, if $(u_1,\ldots,u_n)$ and $(v_1,\ldots,v_n)$ are unimodular rows such that $(u_2,\ldots,u_n,v_2,\ldots,v_n)$ has length $2n-2$, then the conclusion of the Mennicke--Newman lemma holds for these rows as well. This fact will be particularly useful in the proof of Lemma \ref{lem:univ_mn}.
\end{remark}

Let's explain how the preceding two facts allow us to compute van der Kallen's group law on $\mathrm{Um}_n(A)/E_n(A)$. First, the Mennicke--Newman lemma shows that \textit{any} pair of orbits in $\mathrm{Um}_n(A)/E_n(A)$ can be represented by unimodular rows of the form $(a,u_2,\ldots,u_n)$ and $(1-a,u_2,\ldots,u_n)$. Thus, by relation (\ref{eqn:mn_relation})-(\ref{eqn:mn_relation2}), the sum of the corresponding orbits is simply the orbit of the unimodular row $(a(1-a),u_2,\ldots,u_n)$. 

\subsection{The Group Isomorphism}\label{sec:lerbet}

In this section, we prove \cite[Theorem 5.1]{LERBET2024109415}, showing that Fasel's natural bijection 
\[
\mathrm{Um}_n(A)/E_n(A) \xto{\sim} [\Spec A, \A^n_k\setminus\{0\}]_{\A^1}
\]
has the structure of a group homomorphism whenever both sides of the bijection above carry a group structure. The exposition in this section closely follows that of \cite[Section 5]{LERBET2024109415}. In Subsection \ref{sec:group_structure}, we equipped $\mathrm{Um}_n(A)/E_n(A)$ with the structure of an abelian group whenever the stable dimension of $A$ is at most $2n-4$. In particular, this structure also exists when the \textit{Krull dimension} of $A$ is at most $2n-4$. 

\subsubsection{Motivic Cohomotopy Groups}

Let's first demonstrate how to equip $[\Spec A, \A^n_k\setminus\{0\}]_{\A^1}$ with the structure of an abelian group under the same assumptions. Intuitively, since we should think of $\A^n_k\setminus\{0\}$ as a motivic analog of the sphere $S^{n-1}$, this construction is a motivic analog of Borsuk's cohomotopy group structure on $\pi^{n-1}(X)$. This group law is thus aptly called the \textit{motivic Borsuk's group law}. We follow the exposition in \cite[Section 4]{LERBET2024109415}. 

Let $X$ and $Y$ be pointed motivic spaces over $k$, with morphisms $f,g: X \to Y$. These morphisms induce a product map
\[
X \xto{\Delta} X \times X \xto{f \times g} Y \times Y,
\]
where $\Delta: X \to X \times X$ is the diagonal. Our goal is to define a ``sum" of the (classes of the) maps $f$ and $g$ in $[X, Y]_{\A^1}$. In other words, we want an (associative) map 
\[
[X, Y]_{\A^1} \times [X, Y]_{\A^1} \to [X,Y]_{\A^1}. 
\]
We have a natural fold map $Y \vee Y \to Y$, and on the other hand, we have a natural embedding $Y \vee Y \to Y \times Y$. Thus, if the map $f \times g$ were to factor through $Y \vee Y \to Y \times Y$, we could try to define $f + g$ as the following composition
\[
X \xto{\Delta} X \times X \dashrightarrow Y \vee Y \to Y, 
\]
where the dashed right arrow is the map $X \times X \to Y \vee Y$ induced by $f \times g: X \times X \to Y \times Y$. If this scenario were to hold, a straightforward exercise, using the properties of the fold map $Y \vee Y \to Y$, verifies that the assignment above does indeed give $[X,Y]_{\A^1}$ the structure of an abelian group, whose identity element is given by the constant map $X \to Y$ (recall that all motivic spaces are pointed in our discussion). 

The following result of Asok and Fasel \cite[Proposition 1.2.3]{AF22} gives us a condition for which our hopes may hold. Recall that the $\A^1$-cohomological dimension of a smooth $k$-scheme $X$ is the largest positive integer $d$ such that $H^d_{\operatorname{Nis}}(X,\mathcal{F}) = 0$ for any Nisnevich sheaf $\mathcal{F}$ of abelian groups on $X$. 

\begin{theorem}[Asok--Fasel, 2022]\label{thm:af_group}
    Let $Y$ be a pointed motivic space over $k$ that is also defined over some perfect subfield $k' \subset k$. That is, $Y$ is the base change of some $k'$-motivic space $Y'$. Moreover, suppose $n \geq 2$ is a positive integer. If $Y$ is $(n-1)$-$\A^1$-connected, then the morphism $Y \vee Y \to Y \times Y$ induces a bijection
    \[
    [X, Y \vee Y]_{\A^1} \xto{\sim} [X, Y \times Y]_{\A^1}
    \]
    for any smooth $k$-scheme $X$ with cohomological dimension at most $2n-2$. 
\end{theorem}

\begin{proof}
    For the sake of brevity, we omit the proof, referring the reader to \cite[Proposition 4.1]{LERBET2024109415}, for instance, for a proof. The point is to show that the map $f: Y \vee Y \to Y \times Y$ is $\A^1$-$(2n-2)$-connected, at which point \cite[Lemma 2.22]{LERBET2024109415} would give the desired result. The connectedness of $f$ is computed through an application of the Blakers--Massey theorem.
\end{proof}

In fact, the results of Asok and Fasel \cite[Proposition 1.2.5]{AF22} also show that the group structure on $[X,Y]_{\A^1}$ defined in Theorem \ref{thm:af_group} is abelian. While we will not prove this fact here, we record the following corollary for convenience.

\begin{corollary}
    Let $\operatorname{Sm}_k^{\leq d}$ denote the full subcategory of smooth $k$-schemes of cohomological dimension $\leq d$. The assignment $Y \mapsto [-,Y]_{\A^1}$ defines a functor from the full subcategory of pointed motivic spaces satisfying the assumptions of Theorem \ref{thm:af_group} to the category of presheaves on $\operatorname{Sm}_k^{\leq 2n-2}$ valued in abelian groups. 
\end{corollary}

\begin{remark}
    If $Y$ is a pointed motivic space satisfying the assumptions of Theorem \ref{thm:af_group}, the functor $[-,Y]_{\A^1}: \operatorname{Sm}_k^{\leq d} \to \mathsf{Ab}$ is called the \textit{motivic cohomotopy theory defined by $Y$}.
\end{remark}

The upshot of this discussion is that the space $\A^n_k\setminus\{0\}$, with base point $(1,0,\ldots,0)$, satisfies all assumptions of Theorem \ref{thm:af_group} (with some modifications to indices). Indeed, $\A^n_k\setminus\{0\}$ is certainly defined over any subfield of $k$, and we showed earlier this semester that it is $(2n-2)$-$\A^1$-connected. Thus, Theorem \ref{thm:af_group} equips $[X,\A^n_k\setminus\{0\}]$ with the structure of an abelian group for any smooth $k$-scheme $X$ of cohomological dimension at most $2n-4$. Since the cohomological dimension of a ring $A$ is bounded above by its Krull dimension, this group structure also exists when $X$ has \textit{Krull dimension} at most $2n-4$.

\subsubsection{Lerbet's Theorem -- Preliminaries}

With all these pieces in place, let's discuss the proof of the following theorem of Lerbet.

\begin{theorem}[Lerbet, 2024]\label{thm:lerbet}
    Let $A$ be a smooth $k$-algebra with \textit{Krull dimension} $\leq 2n-4$. The natural bijection
    \[
    \mathrm{Um}_n(A)/E_n(A) \xto{\sim} [\Spec A, \A^n_k\setminus\{0\}]_{\A^1}
    \]
    is a group isomorphism, where the left-hand side is equipped with van der Kallen's group structure and the right-hand side is equipped with the cohomotopical group structure. 
\end{theorem}

In particular, Lerbet's theorem provides what he terms as a cohomotopical re-interpretation of van der Kallen's group law. 

We break the proof of this theorem into several more digestible chunks. For concision, we write $Y_n := \A^n \setminus\{0\}$. We give $Y_n$ the structure of a pointed motivic space by the basepoint $(0,\ldots,1)$. Similarly, equip $Y_n \times Y_n$ with the basepoint $(0,\ldots,1,0,\ldots,1)$, so that the induced map $Y_n \vee Y_n \to Y_n \times Y_n$ is pointed. 

\subsubsection{Lerbet's Theorem -- Unimodular Description of Fold Map}\label{subsec:lerbet_unimodular}

The main ingredient in the proof of Theorem \ref{thm:lerbet} is an explicit description of the fold map $Y_n \vee Y_n \to Y_n$ that uses the language of unimodular rows. Let's adopt the notation $U_n := \A^1 \times Y_{n-1} \subset Y_n$ and consider the subscheme
\[
Z_n := (U_n \times Y_n) \cup (Y_n  \times U_n)\subset Y_n \times Y_n.
\]
Here is an elementary but crucial fact about $Z_n$: the map $\iota: Y_n \vee Y_n \to Y_n \times Y_n$ factors through $Z_n \to Y_n \times Y_n$. Indeed, in the first factor, the map $\iota$ is given by $(x_1,\ldots,x_n) \mapsto (x_1,\ldots,x_n,0,\ldots,1) \in Y_n \times U_n \subset Z_n$, and a similar argument can be made for the second factor. 

Our goal now is to construct a map $\pi_n: Z_n \to Y_n$ in the category of motivic spaces such that the fold map $Y_n \vee Y_n \to Y_n$ factors as a composition of the inclusion $Y_n \vee Y_n \to Z_n$ and $\pi_n$. Here is the motivation for this goal. Fix $f,g \in [\Spec A, Y_n]_{\A^1}$ in the category of motivic spaces. As we will soon establish, the product $f \times g: \Spec A \to Y_n \times Y_n$ factors through $Y_n \vee Y_n \to Y_n \times Y_n$, and the cohomotopical group law defines $f + g$ as the composition of $f \times g$ with the fold map $Y_n \vee Y_n \to Y_n$. If we were to construct the map $\pi_n$ described above, then observe that the sum $f + g$ can equivalently be described as the composition $\pi_{n} \circ (f \times g)$. In particular, if we can obtain a concrete description of the map $\pi_n$ in terms of unimodular rows, then we may hope to relate the sum $f + g$ with a sum in van der Kallen's group law. 

To construct $\pi_n$, we construct a Jouanolou device for $Z_n$. First, we produce a Jouanolou device for $Y_{2n-2}$. For any $n \geq 0$, define the smooth integral quadric hypersurfaces
    \begin{align*}
        Q_{2n+1} &:= \Spec k[x_1,\ldots,x_{n+1},y_1,\ldots,y_{n+1}]/\langle x_1y_1 + \cdots + x_{n+1}y_{n+1}- 1\rangle \subset \A^{2n+2}, \\[10pt]
        Q_{2n} &:= \Spec k[x_1,\ldots,x_n,y_1,\ldots,y_n,z]/\langle x_1y_1 + \cdots + x_ny_n + z(1-z)\rangle \subset \A^{2n+1}. 
\end{align*}

\begin{proposition}\cite[Lemma 2.8]{LERBET2024109415} 
    The map $Q_{2n+1} \to Y_{n+1}$ given by projection to the first $n+1$ coordinates is a Jouanolou device.
\end{proposition}

\begin{proof}
    Let $\mathcal{O}$ denote the structure sheaf of $Y_{n+1}$. For fixed $(x_1,\ldots,x_{n+1}) \in Y_{n+1}$, consider the epimorphism
    \[
    \mathcal{O}^{\oplus n+1} \twoheadrightarrow \mathcal{O}, \quad (a_1,\ldots,a_{n+1}) \mapsto 
    a_1x_1 + \cdots + a_{n+1}x_{n+1}.\] The kernel of this morphism is a locally free $\mathcal{O}$-module and thus defines an algebraic vector bundle over $Y_{n+1}$. This vector bundle is precisely $Q_{2n+1} \to Y_{n+1}$.
\end{proof}

We use $Q_{2n+1} \to Y_{n+1}$ to construct a Jouanolou device for $Z_n$. Consider the affine scheme $Z_n'$ defined by the following Cartesian square
% https://q.uiver.app/#q=WzAsNCxbMCwwLCJaX24nIl0sWzAsMSwiUV97NG4tNX0iXSxbMSwwLCJRX3sybi0xfSBcXHRpbWVzIFFfezJuLTF9Il0sWzEsMSwiXFxtYXRoYmJ7QX1eezJuLTJ9Il0sWzAsMV0sWzEsM10sWzIsM10sWzAsMl0sWzAsMywiIiwxLHsic3R5bGUiOnsibmFtZSI6ImNvcm5lciJ9fV1d
\[\begin{tikzcd}
	{Z_n'} & {Q_{2n-1} \times Q_{2n-1}} \\
	{Q_{4n-5}} & {\mathbb{A}^{2n-2}}
	\arrow[from=1-1, to=1-2]
	\arrow[from=1-1, to=2-1]
	\arrow["\lrcorner"{anchor=center, pos=0.125}, draw=none, from=1-1, to=2-2]
	\arrow[from=1-2, to=2-2]
	\arrow[from=2-1, to=2-2]
\end{tikzcd}\]
where the bottom horizontal map is the composition $Q_{4n-5} \to Y_{2n-2} \to \mathbb{A}^{2n-2}$ and the right vertical map is the composition $Q_{2n-1} \times Q_{2n-1} \to Y_{n} \times Y_{n} \to \A^{2n-2}$. Note that $Z_n'$ is affine because it is the fibered product of affine schemes over an affine base. On the other hand, we have a Cartesian square
% https://q.uiver.app/#q=WzAsNCxbMCwwLCJaX24iXSxbMSwwLCJcXG1hdGhiYntBfV5uIFxcdGltZXMgXFxtYXRoYmJ7QX1ebiJdLFsxLDEsIlxcbWF0aGJie0F9Xnsybi0yfSJdLFswLDEsIllfezJuLTJ9Il0sWzMsMl0sWzEsMl0sWzAsM10sWzAsMV0sWzAsMiwiIiwxLHsic3R5bGUiOnsibmFtZSI6ImNvcm5lciJ9fV1d
\[\begin{tikzcd}
	{Z_n} & {\mathbb{A}^n \times \mathbb{A}^n} \\
	{Y_{2n-2}} & {\mathbb{A}^{2n-2}}
	\arrow[from=1-1, to=1-2]
	\arrow[from=1-1, to=2-1]
	\arrow["\lrcorner"{anchor=center, pos=0.125}, draw=none, from=1-1, to=2-2]
	\arrow[from=1-2, to=2-2]
	\arrow[from=2-1, to=2-2]
\end{tikzcd}\]
where the right-vertical map $\A^n \times \A^n \to \A^{2n-2} = \A^{n-1} \times \A^{n-1}$ given by projection to the first $n-1$ coordinates in each factor. Observe that the map $Q_{4n-5} \to \mathbb{A}^{2n-2}$ factors through $Y_{2n-2}$ by definition, so one can show that the composition $Z_n' \to Q_{2n-1} \times Q_{2n-1} \to \mathbb{A}^{2n-2}$ must factor through $Z_n$ (the square above shows that $Z_n$ is the scheme-theoretic preimage of $Y_{2n-2}$ under $\A^n \times \A^n \to \mathbb{A}^{2n-2}$). From this discussion, we see that there is a Cartesian square
% https://q.uiver.app/#q=WzAsNCxbMCwwLCJaX24nIl0sWzEsMCwiWl9uIFxcdGltZXNfe1lfbiBcXHRpbWVzIFlfbn0oUV97Mm4tMX0gXFx0aW1lcyBRX3sybi0xfSkiXSxbMCwxLCJaX25cXHRpbWVzX3tZX3sybi0yfX1RX3s0bi01fSJdLFsxLDEsIlpfbiJdLFswLDJdLFsyLDNdLFsxLDNdLFswLDFdXQ==
\[\begin{tikzcd}
	{Z_n'} & {Z_n \times_{Y_n \times Y_n}(Q_{2n-1} \times Q_{2n-1})} \\
	{Z_n\times_{Y_{2n-2}}Q_{4n-5}} & {Z_n}
	\arrow[from=1-1, to=1-2]
	\arrow[from=1-1, to=2-1]
	\arrow[from=1-2, to=2-2]
	\arrow[from=2-1, to=2-2]
\end{tikzcd}\]
The bottom and right maps in the square above are vector bundles because they are base changes of the Jouanolou devices $Q_{4n-5}\to Y_{2n-2}$ and $Q_{2n-1} \times Q_{2n-1} \to Y_n \times Y_n$, respectively. Hence, the top and left maps are also vector bundles. We deduce that either composition $Z_n' \to Z_n$ in the diagram above is a Jouanolou device for $Z_n$. Let's return to the construction of the map $\pi_n$. Since $Z_n' \to Z_n$ induces a motivic equivalence, it suffices to study maps $Z_n' \to Y_n$. 

The key reason that we want to consider the scheme $Z_n'$ is due to its explicit connection to unimodular rows. Lerbet \cite[pg. 27]{LERBET2024109415} describes the  coordinate ring of $Z_n'$ as the quotient 
\[
k[Z_n'] = k[x_1,\ldots,x_ny_1,\ldots,y_n,u_1,\ldots,u_n,v_1,\ldots,v_n,r_2,\ldots,r_n,s_2,\ldots,s_n]/I
\]
where $I$ is the ideal generated by relations
\begin{align*}
    x_1y_1 + x_2y_2 + \cdots + x_ny_n &= 1 \\
    u_1v_1 + u_2v_2 + \cdots + u_nv_n &= 1 \\
    x_2r_2 + \cdots + x_nr_n + u_2s_2 + \cdots + u_ns_n &= 1.
\end{align*}
As Lerbet remarks, we should think of $(x_1,\ldots,x_n,u_1,\ldots,u_n)$ as the universal vector such that $(x_1,\ldots,x_n)$, $(u_1,\ldots,u_n)$, and $(x_2,\ldots,x_n,u_2,\ldots,u_n)$ are unimodular. In other words, for any $k$-algebra $R$ and any morphism $R^{2n+2} \to R$ given by a vector $(x_1',\ldots,x_n',u_1',\ldots,u_n')$ satisfying the conditions above, there exists a unique map $k[Z_n'] \to R^{2n+2}$ given by $x_i \mapsto x_i'$ and $u_i \mapsto u_i'$. This observation will prove crucial when we proceed to the proof of Theorem \ref{thm:lerbet}.

\begin{lemma}\label{lem:univ_mn}
    There exist $\gamma,a_2,\ldots,a_n \in k[Z_n']$ and elementary matrices $E_x,E_u \in E_n(k[Z_n'])$ such that $(x_1,\ldots,x_n)E_x = (\gamma,a_2,\ldots,a_n)$ and $(u_1,\ldots,u_n)E_u = (1-\gamma,a_2,\ldots,a_n)$.
\end{lemma}

\begin{proof}
    We should think of this lemma as a ``universal" analogue of the Mennicke--Newman Lemma \ref{lem:mn}, as $(x_1,\ldots,x_n)$ and $(u_1,\ldots,u_n)$ are unimodular rows. More precisely, we are almost in the situation of the Mennicke--Newman lemma, except we do not have the needed dimensionality hypotheses on $k[Z_n']$. We circumvent this issue by Remark \ref{rem:no_dim_assumption}. That is, observe that the dimensionality hypothesis in the proof of the Mennicke--Newman lemma is used only when proving that $(x_2,\ldots,x_n,u_2,\ldots,u_n)$ is also unimodular. However, thanks to the defining relations of $k[Z_n']$, we automatically have that $(x_2,\ldots,x_n,u_2,\ldots,u_n)$ is unimodular. Hence, the rest of the proof of the Mennicke--Newman lemma applies, giving us the result.
\end{proof}

In particular, thanks to the relation (\ref{eqn:mn_relation})-(\ref{eqn:mn_relation2}) for unimodular rows, we see that $(\gamma(1-\gamma),a_2,\ldots,a_n)$ is also a unimodular row for $k[Z_n']$, and hence, we obtain a morphism $\pi_n': Z_n' \to Y_n$ of schemes. Since $Z_n' \to Z_n$ is a motivic equivalence, the map $\pi_n'$ uniquely determines a morphism $Z_n \to Y_n$ in the category of motivic spaces. It remains to describe the fold map $Y_n \vee Y_n \to Y_n$ in terms of $\pi_n$. 

\begin{proposition}\cite[Lemma 5.8]{LERBET2024109415}\label{prop:big_upshot}
    There is a commutative diagram in the category of pointed motivic spaces
    % https://q.uiver.app/#q=WzAsMyxbMCwwLCJZX24gXFx2ZWUgWV9uIl0sWzEsMSwiWV9uIl0sWzEsMCwiWl9uIl0sWzAsMl0sWzIsMSwiXFxwaV9uIl0sWzAsMV1d
\[\begin{tikzcd}
	{Y_n \vee Y_n} & {Z_n} \\
	& {Y_n}
	\arrow[from=1-1, to=1-2]
	\arrow[from=1-1, to=2-2, "{\nabla}"]
	\arrow["{\pi_n}", from=1-2, to=2-2]
\end{tikzcd}\]
where the diagonal map is the fold map $\nabla: Y_n \vee Y_n \to Y_n$ and the horizontal map is the inclusion $Y_n \vee Y_n \to Z_n$.
\end{proposition}

\begin{proof}
    Refer to the proof of \cite[Lemma 5.8]{LERBET2024109415}, relying entirely on elementary unimodular row operations.
\end{proof} 

\subsubsection{Lerbet's Theorem -- Motivic Ingredients}\label{subsec:lerbet_motivic}

The upshot of our work in the preceding subsubsection is quite substantial. To rigorously formulate the consequences, we record some additional motivic facts. 

\begin{lemma}\label{lem:easy}
    Let $A$ be a smooth commutative $k$-algebra with Krull dimension at most $2n-4$. The inclusion $Y_n \vee Y_n \to Y_n \times Y_n$ induces a bijection $[\Spec A, Y_n \vee Y_n]_{\A^1} \to [\Spec A, Y_n \times Y_n]_{\A^1}$. 
\end{lemma}

\begin{proof}
    An immediate consequence of Theorem \ref{thm:af_group}, since $Y_n$ is $(n-1)$-$\A^1$-connected and the cohomological dimension of $A$ is bounded by its Krull dimension.
\end{proof}

We would like to prove a similar result for the inclusion $Z_n \to Y_n \times Y_n$. We need a sequence of lemmas to achieve this result.

\begin{lemma}\cite[Lemma 4.5]{LERBET2024109415}
    If $n \geq 2$, then $Z_n$ is $\A^1$-connected. 
\end{lemma}

\begin{proof}
    We provide a sketch of the proof, referring the reader to \cite[Lemma 5.5]{LERBET2024109415} for details. First, one shows that $Y_n = \A^n \setminus \{0\}$ is $\A^1$-chain connected for $n \geq 2$. That is, for any finite separable extension $L/k$ and any points $x,x' \in Y_n(L)$, there exists a morphism $g: \A^1_L \to Y_n(L)$ such that $g(0) = x$ and $g(1) = x'$. This step is elementary and explicit. Next, the product of $\A^1$-chain connected schemes is also $\A^1$-chain connected, so $U_n \times Y_n$ and $Y_n \times U_n$ are both $\A^1$-chain connected. Finally, the union $Z_n = U_n \times Y_n \cup Y_n \times U_n$ is $\A^1$-chain connected because the schemes $U_n \times Y_n$ and $Y_n \times U_n$ share a base point. Since $\A^1$-chain connectedness implies $\A^1$-connectedness, the result follows.
\end{proof}

\begin{lemma}\cite[Lemma 5.6]{LERBET2024109415}
    Suppose $n \geq 3$. The inclusion $Z_n \to Y_n \times Y_n$ is $\A^1$-$(2n-4)$-connected.
\end{lemma}

\begin{proof}
    The first step of the proof is to show that $Z_n$ is $\A^1$-simply connected. We already kow that $Z_n$ is $\A^1$-connected. On the other hand, we have \[\pi_1^{\A^1}(U_n \times Y_n) = \pi_1^{\A^1}(Y_n) \times \pi_1^{\A^1}(Y_{n-1}) \times \pi_1^{\A^1}(\A^1) = \pi_1^{\A^1}(Y_n) \times \pi_1^{\A^1}(Y_{n-1})= 1\] because $Y_m = \A^m\setminus\{0\}$ is $\A^1$-simply connected for $m \geq 2$. Similalry, $\pi_1^{\A^1}(Y_n \times U_n)$ is trivial. On the other hand, the intersection $(Y_n \times U_n) \cap (U_n \times Y_n) = U_n \times U_n$ is also $\A^1$-simply connected. We conclude with an application of the motivic van Kampen theorem cited by Lerbet in \cite[Lemma 5.6]{LERBET2024109415}. 

    Next, Lerbet explains that we have a cofiber sequence
    \[
    Z_n \to Y_n \times Y_n \to (\G_m \times \G_m)_+ \wedge (\P^1)^{\wedge 2n-2},
    \]
    so the homotopy cofiber of the map $Z_n \to Y_n \times Y_n$ is $\A^1$-$(2n-3)$-connected. From here, the result follows from an application of the Blakers--Massey theorem also cited by Lerbet. We omit the details for the sake of brevity and refer the reader to \cite[Lemma 5.6]{LERBET2024109415}.
\end{proof}

A standard motivic obstruction theory result \cite[Lemma 2.22]{LERBET2024109415} allows us to conclude the desired

\begin{corollary}\label{cor:easy}
    The map $Z_n \to Y_n \times Y_n$ induces a bijection $[\Spec A, Z_n]_{\A^1} \to [\Spec A, Y_n \times Y_n]_{\A^1}$.
\end{corollary}

Let's discuss the consequences of this subsubsection and subsubsection \ref{subsec:lerbet_unimodular}. Fix morphisms $f,g: \Spec A \to Y_n$ and define $[f,g]: \Spec A \to Y_n \times Y_n$ as the composition $(f\times g) \circ \Delta$, where $\Delta: \Spec A \to \Spec A \times \Spec A$ is the diagonal embedding. Recall that the morphism $Y_n \vee Y_n \to Y_n \times Y_n$ factors through the inclusion $Z_n \to Y_n \times Y_n$. By Lemma \ref{lem:easy} and Corollary \ref{cor:easy}, there exists unique lifts $(f,g): \Spec A \to Y_n \vee Y_n$ and $\langle f,g\rangle: \Spec A \to Z_n$ so that the following diagram commutes
% https://q.uiver.app/#q=WzAsNCxbMSwwLCJaX24iXSxbMSwxLCJZX24gXFx2ZWUgWV9uIl0sWzAsMiwiXFxvcGVyYXRvcm5hbWV7U3BlY31BIl0sWzIsMiwiWV9uIFxcdGltZXMgWV9uIl0sWzAsM10sWzEsMF0sWzEsM10sWzIsMCwie1xcbGFuZ2xlIGYsZ1xccmFuZ2xlfSIsMV0sWzIsMSwieyhmLGcpfSIsMV0sWzIsMywie1tmLGddfSIsMV1d
\[\begin{tikzcd}
	& {Z_n} \\
	& {Y_n \vee Y_n} \\
	{\operatorname{Spec}A} && {Y_n \times Y_n}
	\arrow[from=1-2, to=3-3]
	\arrow[from=2-2, to=1-2]
	\arrow[from=2-2, to=3-3]
	\arrow["{{\langle f,g\rangle}}"{description}, from=3-1, to=1-2]
	\arrow["{{(f,g)}}"{description}, from=3-1, to=2-2]
	\arrow["{{[f,g]}}"{description}, from=3-1, to=3-3]
\end{tikzcd}\]
The sum $f + g \in [\Spec A,Y_n]_{\A^1}$ in the cohomotopical group law is defined as the composition $f+g := \nabla \circ (f,g)$, where $(f,g): \Spec A \to Y_n \vee Y_n$ is the unique lift of the composition $(f \times g) \circ \Delta: \Spec A \to Y_n \times Y_n$. By Proposition \ref{prop:big_upshot} and the uniqueness of the lift $\langle f,g\rangle$, the following diagram commutes:
% https://q.uiver.app/#q=WzAsNCxbMCwxLCJcXG9wZXJhdG9ybmFtZXtTcGVjfUEiXSxbMiwxLCJZX24gXFx2ZWUgWV9uIl0sWzIsMCwiWl9uIl0sWzQsMSwiWV9uIl0sWzAsMiwiXFxsYW5nbGUgZixnXFxyYW5nbGUiLDFdLFswLDEsIihmLGcpIiwxXSxbMSwzLCJcXG5hYmxhIiwxXSxbMSwyXSxbMiwzLCJcXHBpX24iLDFdXQ==
\[\begin{tikzcd}
	&& {Z_n} \\
	{\operatorname{Spec}A} && {Y_n \vee Y_n} && {Y_n}
	\arrow["{\pi_n}"{description}, from=1-3, to=2-5]
	\arrow["{\langle f,g\rangle}"{description}, from=2-1, to=1-3]
	\arrow["{(f,g)}"{description}, from=2-1, to=2-3]
	\arrow[from=2-3, to=1-3]
	\arrow["\nabla"{description}, from=2-3, to=2-5]
\end{tikzcd}\]
Thus, we can also compute $f + g$ as the composition \[f + g = \pi_n \circ \langle f,g \rangle.\] Since $\pi_n$ has an explicit description in terms of a Mennicke--Newman-like relation for unimodular rows, we can now expect some relationship between the cohomotopical and the van der Kallen group laws on $[\Spec A, X]_{\A^1}$.

\subsubsection{Lerbet's Theorem -- Proof}

It remains to prove the following assertion.

\begin{proposition}\label{prop:almost_there}
    Suppose $A$ is a smooth $k$-algebra of Krull dimension at most $2n-4$. The natural map $\varphi: \operatorname{Um}_n(A)/E_n(A) \to [\Spec A, Y_n]_{\A^1}$ is a group homomorphism.
\end{proposition}
\begin{proof}
    Take $[u],[u'] \in \operatorname{Um}_n(A)/E_n(A)$. Thanks to the Mennicke--Newman lemma, we may assume that there representatives have the form $u = (x,a_2,\ldots,a_n)$ and $u' = ((1-x),a_2,\ldots,a_n)$ for some $x,a_2,\ldots,a_n \in A$. Van der Kallen's group law then gives us
    \[
    [u] + [u'] = [(x(1-x),a_2,\ldots,a_n)].
    \]
    For any unimodular row $(v_1,\ldots,v_n)$, the morphism $\varphi(v): \Spec A \to Y_n$ is the $A$-point \[(v_1,\ldots,v_n) \in Y_n(A) = \A^n(A)\setminus\{0\}.\] 
    Thus, the unimodular rows $u$ and $u'$ jointly define the morphism 
    \[
    \langle\varphi(u),\varphi(u')\rangle: \Spec A \to Z_n
    \]
    given by the $A$-point
    \[(x,a_2,\ldots,a_n,1-x,a_2,\ldots,a_n) \in Z_n \subset Y_n \times Y_n.\]
    The map $\pi_n$ was defined so that
    \[\pi_n(x,a_2,\ldots,a_n,1-x,a_2,\ldots,a_n) = (x(1-x),a_2,\ldots,a_n) = \varphi([u] + [u']).\]
    The left-hand side is the composition $\pi_n \circ \langle \varphi(u),\varphi(u')\rangle$. At the end of subsubsection \ref{subsec:lerbet_motivic}, we noted that this composition is precisely the sum $\varphi([u]) + \varphi([u'])$ in the cohomotopical group law, so we are done.
\end{proof}

Without further ado, we can wrap up the proof of Theorem \ref{thm:lerbet}.

\begin{proof}[Proof of Theorem \ref{thm:lerbet}]
    A direct consequence of Theorem \ref{thm:fasel} and Proposition \ref{prop:almost_there}.
\end{proof}

\subsection{Cohomological Interpretation of Unimodular Rows}\label{sec:cohomology}

We conclude this appendix with a discussion of some other results on unimodular rows explored from a motivic perspective. First, we discuss Fasel's cohomological interpretation of the group structure on $\mathrm{Um}_n(A)/E_n(A)$ (i.e., Theorem \ref{thm:fasel_cohomology}). Following Fasel, we will give an explicit computation of this group for some examples of $\R$-algebras. We concludediscuss some related applications of this computation to the study of stably free modules, per Morel and Fasel. 

\subsubsection{Cohomology and Unimodular Rows} In contrast with Lerbet's geometric interpretation van der Kallen's group structure, Fasel offers a cohomological perspective on the group $\operatorname{Um}_n(A)/E_n(A)$, though only for $n = \dim(A) + 1$. In this subsubsection, we draw our material from \cite[Sections 3, 4]{Fasel_2010}. His results relate this group to the cohomology of the Milnor--Witt sheaf on $\Spec A$.

\begin{theorem}\cite[Theorem 4.9]{Fasel_2010}
    Let $k$ be a perfect field of characteristic not equal to two, and suppose $A$ is a smooth $k$-algebra of Krull dimension $d = n-1 \geq 2$. There exists a natural isomorphism 
    \[
    \mathrm{Um}_n(A)/E_n(A) \xto{\sim} H^{n-1}(\Spec A,K_{n}^{MW}),
    \]
    where $K_*^{MW}$ is the Milnor--Witt sheaf.
\end{theorem}

\begin{remark}
    Strictly speaking, Fasel proves the result above for a sheaf $G^{n}$ of abelian groups instead of $K_*^{MW}$. However, as he remarks in the introduction to his paper, we have isomorphisms of cohomology groups $H^{n-1}(\Spec A, K_n^{MW}) \cong H^{n-1}(\Spec A, G^n)$ (using the notation above) whenever $\operatorname{char}(k) \neq 2$.
\end{remark}

Instead of providing the full proof of the theorem above, which requires a long technical detour, we simply direct the reader to \cite[Section 4]{Fasel_2010} for details. Instead, let's give a brief description of how the homomorphism in the theorem is constructed. Given a smooth $k$-algebra $A$, Fasel produces a map
\[
\varphi: \mathrm{Um}_n(A)/E_n(A) \to H^{n-1}(\Spec A,K_{n}^{MW})
\]
as follows. We have a natural map
\[
\Hom_{k\mathrm{-Sch}}(\Spec A, \A^{m+1} \setminus\{0\}) \to H^{m}(A,K_{m+1}^{MW})
\]
given by $f \mapsto f^*(\xi)$, where $\xi \in H^{m}(\A^m\setminus\{0\},K_{m+1}^{MW})$ is some distinguished class defined in \cite[Section 3]{Fasel_2010}. One can show that this map is $\A^1$-naive homotopy invariant, so that it factors through $\operatorname{Um}_n(A)/E_n(A)$ by Proposition \ref{prop:fasel_naive}. The fact that this map is a homomorphism is the content of the proof of \cite[Theorem 4.1]{Fasel_2010}. There, Fasel notes that it suffices to verify the relation (\ref{eqn:mn_relation})-(\ref{eqn:mn_relation2}) in $H^{m}(A,K_{m+1}^{MW})$ and reduces the verification of this relation to a simple computation using the defining relations of the Milnor--Witt group $K_1^{MW}(k(t))$. The remainder of \cite[Section 4]{Fasel_2010} is dedicated to proving that this homomorphism is an isomorphism when $n-1 = \dim(A) \geq 2$ (in fact, the cases $\dim(A) = 2,3$ must be handled separately), and we refer the reader to that section for details of this proof, which are quite technical and perhaps beyond the scope of this appendix.

As a consequence of this theorem, however, Fasel is able to explicitly compute the group $\mathrm{Um}_n(A)/E_n(A)$ in some special cases. He works over the field $\R$ in particular. Recall that a smooth $\R$-variety $X$ is called rational if the base change to $\C$ is birational to $\P_\C^d$ for some $d$. As a result of some rather technical computations (we refer the reader to \cite[Section 5]{Fasel_2010} for details), Fasel proves the following result. In particular, under some restrictive assumptions, he shows that the group $\operatorname{Um}_n(A)/E_n(A)$ is free abelian with an explicitly determined indexing set.

\begin{theorem}\cite[Theorem 5.7, Remark 5.8]{Fasel_2010}\label{thm:fasel_explicit_group}
    Suppose $A$ is a smooth $\R$-algebra of Krull dimension $d = n-1 \geq 2$. Moreover, assume that $X = \Spec(A)$ is rational and has trivial canonical bundle. Then, we have an isomorphism
    \[
    \operatorname{Um}_n(A)/E_n(A) \cong H^{n-1}(X, K_{n}^{MW}) \cong \Z^{\oplus \pi_0^c(X(\R))}
    \]
    where $\pi_0^c(X(\R))$ denotes the set of compact connected components of the Euclidean space $X(\R)$. In particular, if $d \geq 3$, we have an isomorphism
    \[
    \operatorname{Um}_n(A)/E_n(A) \cong \pi^d(X(\R)),
    \]
    where $\pi^d(X(\R))$ is the $d$th cohomotopy group.
\end{theorem}

\subsubsection{Applications to Stably Free Modules} In this section, we discuss some applications of the motivic perspective on unimodular rows to the study of stably free modules, following the work of Fasel \cite[Section 5.2]{Fasel_2010}. Recall that an $A$-module $M$ is called \textit{stably free} if there exists some $r \geq 0$ such that $M \oplus A^{\oplus r}$ is a finite rank free $A$-module. Observe that a stably free module is necessarily projective. The study of stably free modules arises as a special case of the following question in commutative algebra.

\begin{question}
    Let $P$ and $Q$ be projective modules such that $P \oplus A^{\oplus n} \cong Q \oplus A^{\oplus n}$ for some $n \geq 0$. Are $P$ and $Q$ isomorphic?
\end{question}

This question finds a partial resolution through a result of Bass and Schanuel \cite[Theorem 2]{Bass_Schanuel}:

\begin{theorem}(Bass--Schanuel, 1962)
    If $P$ is a finitely generated projective $A$-module of $f$-rank at least the Krull dimension of $A$, and $Q$ is any finitely generated projective $A$-module satisfying $P \oplus A^{\oplus r} \cong Q \oplus A^{\oplus r}$ for some $r \geq 0$, then $P \cong Q$.
\end{theorem}

Observe that the Bass--Schanuel theorem reduces the study of stably free modules to the special case of finitely generated projective modules $P$ such that $P \oplus A \cong A^{\oplus n}$, where $n-1$ is the Krull dimension of $A$. It turns out that the study of these modules can be handled with the theory of unimodular rows, and this source of motivation actually was the impetus for Fasel's study of unimodular rows in \cite{Fasel_2010}. 

Indeed, we claim that a finitely generated projective $A$-module $P$ satisfying $P \oplus A \cong A^{\oplus n}$ for some $n \geq 0$ is the same data as a unimodular row $u \in \operatorname{Um}_n(A)$. Given a unimodular row $(u_1,\ldots,u_n)$, we take $P \subset A^{\oplus n}$ as the kernel of the corresponding surjection $A^{\oplus n} \to A$. Note that the resulting short exact sequence \[0 \to P \to A^{\oplus n} \to A \to 0\] splits, as desired. Conversely, a split short exact sequence as above contains a surjective homomorphism $A^{\oplus n} \to A$, which gives us a unimodular row of length $n$. Of course, note that isomorphisms of the modules $P$ correspond to equivalences of unimodular rows under the natural action of $\mathrm{GL}_n(A)$. Thus, we are actually interested in the orbit set \[\operatorname{Um}_n(A)/\mathrm{GL}_n(A) \simeq \mathrm{Um}_n(A)/\mathrm{SL}_n(A),\] a quotient of $\mathrm{Um}_n(A)/E_n(A)$. Given the results of Fasel that we presented the previous subsubsection, it is perhaps unsurprising that Morel\footnote{I am unable to find the reference cited by Fasel's paper} found a cohomological interpretation of the  orbit set above. Strictly speaking, Morel's result only applies to smooth algebras $A$ of Krull dimension at least three. Fasel \cite[Theorem 4.11]{Fasel_2010} extends Morel's result to the dimension two case, resulting in the following theorem.

\begin{theorem}[Fasel--Morel]
    Let $A$ be a smooth $k$-algebra of Krull dimension $d = n-1 \geq 2$ over a perfect field $k$ with characteristic not equal to two. There is a natural bijection 
    \[
    \mathrm{Um}_n(A)/\mathrm{SL}_n(A) \cong H^{n-1}(\Spec A, K^{MW}_n)/\mathrm{SL}_n(A).
    \]
\end{theorem}

The proof of this theorem relies on the study of some short exact sequences related to the Chow--Witt group. In order to avoid a long technical discussion, we omit the proof and refer the reader to \cite[Section 4]{Fasel_2010}. Similar to the case of elementary orbits of unimodular rows, Fasel uses the theorem above to explicitly compute the set $\mathrm{Um}_n(A)/\mathrm{SL}_n(A)$ in the case of some exceptional real algebras $A$.

\begin{theorem}\cite[Theorem 5.9]{Fasel_2010}\label{thm:fasel_stable}
    Let $A$ be a smooth $\R$-algebra of even Krull dimension $d$ such that $\Spec A$ has trivial canonical bundle and is rational. The set of isomorphism classes of stably free $A$-modules is isomorphic to $\Z^{\oplus \pi_0^c(X(\R))}$, where $\pi_0^c(X(\R))$ is the set of compact connected components of $X(\R)$.
\end{theorem}

The proof of this theorem crucially uses Theorem \ref{thm:fasel_explicit_group} and thus the cohomological/motivic machinery employed by Fasel. As a consequence of Theorem \ref{thm:fasel_stable}, Fasel gives a criterion for some stably free modules to be free. 

\begin{theorem}\cite[Theorem 5.10]{Fasel_2010}
    Suppose $A$ is an $\R$-algebra satisfying the assumptions of Theorem \ref{thm:fasel_stable}. A stably free $A$-module of rank $\dim(A)$ is free if and only if its Euler class vanishes.
\end{theorem}

Fasel's classification result above finds the following curious application. 

\begin{corollary}\cite[Corollary 5.12]{Fasel_2010}
    The set of isomorphism classes of stably free rank $2d$ modules on the sphere $S^{2d}$ (i.e., stably free vector bundles on $S^{2d}$) is isomorphic to $\Z$ and is generated by the tangent bundle. 
\end{corollary}
