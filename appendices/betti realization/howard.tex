\providecommand{\Be}{\mathrm{Be}}
\providecommand{\Loop}{\Omega}
\providecommand{\Smash}{\wedge}
\providecommand{\Sph}{\mathbb{S}}
\providecommand{\triv}{\mathrm{triv}}
\providecommand{\isom}{\simeq}
\providecommand{\GeomFix}{\Phi}

\providecommand{\SHRcell}{\SH_{\mathrm{cell}} (\R)}
\providecommand{\SpCt}{\Sp^{C_2}}
\providecommand{\BeCt}{\Be^{C_2}}
\providecommand{\pcomp}{_p^{\wedge}}
\providecommand{\pBeCt}{\widehat{\Be}_p^{C_2}}
\providecommand{\pSHRcell}{\SHRcell\pcomp}
\providecommand{\pSpCt}{(\SpCt)\pcomp}

\section{Howard Beck: Real Betti realization}
In this section, we cover an interesting connection between motivic homotopy theory and $ C_2 $-equivariant homotopy theory, following the works of \cite{Bachmann-betti} and \cite{BehrensShah-C2betti}.
We will assume comfort with the language and notation of equivariant (stable) homotopy theory -- for a somewhat old-school but still sufficient (and way too extensive for our purposes here) reference, see \cite{GreenleesMay}.

\subsection{Introduction}
We start with the Betti realization functor, which associates to a smooth scheme over $ \C $ a topological space given by its $ \C $-points, endowed with the complex analytic topology:
\begin{align*}
\Be: \Sm_\C &\to \Top \\
Z &\mapsto Z(\C)
\end{align*}
We may similar form a $ C_2 $-Betti realization, that takes a smooth scheme over $ \R $ and sends it also to its $ \C $-points.
However, here we get a $ C_2 $-action via complex conjugation, so we actually get a map into $ C_2 $-equivariant topological spaces:
\begin{align*}
\BeCt: \Sm_\R &\to \Top^{C_2} \\
Z &\mapsto \{C_2 \curvearrowright Z(\C) \}
\end{align*}

We will soon pass to the stable setting, but we should first fix notation.
We formed the stable motivic homotopy category $ \SH(\R) $ by inverting the loop functor $ \Loop^{2, 1} $ corresponding to $ \P_1 $ which has the effect of inverting all loop functors $ \Loop^{i, j} $ with respect to all motivic spheres $ S^{i, j} $.
Equivariantly, we form genuine $ C_2 $-spectra by inverting loop functors for all representation spheres $ S^\rho $, where $ \rho $ is a finite-dimensional, continuous, real, orthogonal representation of $ C_2 $.
We may similarly only invert the loop functor $ \Loop^{\triv + \sigma} $, where $ \triv $ is the one-dimensional trivial representation and $ \sigma $ is the sign representation.
Doing so will also invert all loop functors, which are given by $ \Loop^{(\triv)i + (\sigma)j} $.
The upshot is that we may view motivic spectra in $ \SH(\R) $ or equivariant spectra in $ \SpCt $ as being either mono- or bi-graded.
\\

Our convention will be to take the bi-graded approach, and then have to make a choice about indexing.
It is not difficult to see that we may take $ C_2 $-Betti realization on the space level of our spectra, and we get a functor:
$$ \BeCt: \SH(\R) \to \SpCt $$
Using our convention for bigrading of motivic spheres, $ C_2 $-Betti realization will send the motivic sphere spectra $ \Sph^{i, j} $ to the genuine $ C_2 $-equivariant sphere spectra $ \Sph_{C_2}^{(i-j)\triv + j \sigma} $.
Therefore, we will choose to the $ \RO(C_2) $ bigrading where $ (i, j) $ corresponds to $ (\triv)i + (\sigma - \triv)j $, since we allow virtual representations such as $ \sigma - 1 $.
\\

We have an inclusion map $ S^{0, 0} = {\pm 1} \hookrightarrow \Gm = S^{1, 1} $.
By stabilizing and desuspending in both dimensions, we get a map $ \rho: \Sph^{-1, -1} \to \Sph^{0, 0} $.
Work by \cite{Bachmann-betti} showed that $ C_2 $-Betti realization acts like $ \rho $-localization:
\begin{theorem}[\cite{Bachmann-betti}, Proposition 31 -- as interpreted by \cite{BehrensShah-C2betti}, Theorem 1.5]
$ \BeCt $ induces an equivalence of $ \infty $-categories $ \SH(\R)[\rho^{-1}] \isom \Sp $:
\[\begin{tikzcd}
	{\SH(\R)} & {\Sp^{\GeomFix C_2} \isom \Sp} \\
	{\SH(\R) [\rho^{-1}]}
	\arrow["{\BeCt}", from=1-1, to=1-2]
	\arrow["{\text{loc}}"', from=1-1, to=2-1]
	\arrow["\simeq", from=2-1, to=1-2]
\end{tikzcd}\]
\end{theorem}
Our goal is to find a similar statement that localizes $ \SH(\R) $ into $ \SpCt $.
This will be slightly too much to ask for, but we do get such a phenomenon when we $ p $-complete everything and restrict ourselves to cellular motivic spectra.
By $ X $ cellular -- or $ X \in \SHRcell $ -- we mean that we may construct $ X $ using motivic sphere spectra via cofiber sequences and filtered homotopy colimits.
We may then take $ p $-completed Betti realization:
\[\begin{tikzcd}
	\pSHRcell & \SpCt && \pSpCt
	\arrow["\BeCt", from=1-1, to=1-2]
	\arrow["\pBeCt"', curve={height=12pt}, from=1-1, to=1-4]
	\arrow["{p\text{-completion}}", from=1-2, to=1-4]
\end{tikzcd}\]
The main result of \cite{BehrensShah-C2betti} is the following:
\begin{theorem}[\cite{BehrensShah-C2betti}, Theorem 1.12]
$ p $-completed cellular $ C_2$-Betti realization -- $ \pBeCt $ -- is a localization functor.
\end{theorem}