\subsection{Cofinality}
\begin{definition}\label{def:cofinal} 
We say a subcategory $I \subseteq J$ is \textit{cofinal} if for any functor $F \colon J \to \mathscr{C}$, the induced map on colimits
\begin{equation}\label{eqn:cofinality}
\begin{aligned}
    \colim_I F \to \colim_J F
\end{aligned}
\end{equation}
is an isomorphism.\footnote{%
We dually say $I \subseteq J$ is \textit{final} if the natural map $\lim_I F \to \lim_J F$ is an isomorphism for any $F$.%
}
In other words, in order to compute a $J$-shaped colimit, it suffices to restrict to the subdiagram $I \subseteq J$.
\end{definition}

\begin{remark} This definition makes sense in 1-category theory as well as it does in $\infty$-category theory, however we remark that the two notions are different, so let's differentiate between the two, continuing the story of \autoref{def:cofinal}:
\begin{itemize}
    \item $I \subseteq J$ is $1$\textit{-cofinal} if for any $1$-category $\mathscr{C}$ and $1$-functor $F \colon J \to \mathscr{C}$, the induced map \autoref{eqn:cofinality} is an isomorphism.
    \item $I \subseteq J$ is \textit{cofinal} ($\infty$-cofinal if we want to be really pedantic) if for any $\infty$-category and $\infty$-functor $F \colon J \to \mathscr{C}$, the induced map \autoref{eqn:cofinality} is an equivalence.
\end{itemize}
Note this latter definition extends to the case where $I$ and $J$ are themselves $\infty$-categories, or even just simplicial sets.
\end{remark}

\begin{example}\label{exa:even-odd-cofinal}
The subcategories $2\mathbb{N} \subseteq \mathbb{N}$ and $2 \mathbb{N} + 1 \subseteq \mathbb{N}$ are both 1-cofinal and $\infty$-cofinal. 
\end{example}


It is straightforward to check when a subdiagram is 1-cofinal:
\begin{proposition} \cite[04E6]{Stacks}
$I \subseteq J$ is $1$-cofinal if
\begin{itemize}
    \item every $j\in J$ has some $i \in I$ with a morphism $j \to i$
    \item for every $j \in J$ and pair of objects $i,i'\in I$, there is a zig-zag of morphisms between $i,i'\in I$ and maps from $j$ into the zig-zag making the diagram commute:
\[ \begin{tikzcd}
     &  & j\dar\ar[dr]\ar[dl] &  & \\
    \cdots\rar & i_n & i_{n+1}\lar\rar & i_{n+2} & \lar\cdots \\
\end{tikzcd} \]
\end{itemize}
\end{proposition}

\begin{notation} Let $\DDelta^\inj \subseteq \DDelta$ be the subcategory of injective maps, and let $\DDelta_{\le n} \subseteq \DDelta$ denote the full subcategory of objects $[k]$ for $k\le n$. For example $\DDelta^\inj_{\le 1}$ is just a the category with two parallel arrows, i.e. the ``(co)equalizer'' category:
\begin{align*}
    \DDelta^\inj_{\le 1} := \bullet \rightrightarrows \bullet
\end{align*}
\end{notation}


\begin{example}\label{exa:1-cofinal-parallel-arrows-ddelta} 
We have that $\DDelta_{\le1}^{\inj,\,\op} \subseteq \DDelta^\op$ is 1-cofinal (c.f. \cite[8.3.8]{Riehl}).
\end{example}
% \begin{proof} It's easier to check the dual properties hold before taking the opposite category. It's clear both $[0]$ and $[1]$ map to any element in $\DDelta$. For the second condition we have to check the zig-zag condition. If $i=i'$ in this setting, then it is clear
% \end{proof}

\begin{proposition} The inclusion $\DDelta_{\le n}^{\inj,\op} \subseteq \DDelta^\op$ is $n$-cofinal for any $1\le n \le \infty$.
\end{proposition}


It turns out by a souped-up extension of Quillen Theorem A, originally due to Joyal, we have a necessary condition for cofinality in the $\infty$-categorical setting.

\begin{theorem} (Joyal, Quillen) Let $f \colon I \to J$ be a functor of 1-categories which is ($\infty$-)cofinal. Then the induced map on classifying spaces
\begin{align*}
    BI \to BJ
\end{align*}
is a weak homotopy equivalence \cite[4.1.3.1,~4.1.3.3]{HTT}.
\end{theorem}


\begin{example} The same subcategory $\DDelta_{\le1}^{\inj,\,\op} \subseteq \DDelta^\op$ is \textit{not} $\infty$-cofinal.
\end{example}
\begin{proof} The classifying space of the coequalizer diagram is $S^1$, however since $[0] \in\DDelta$ is terminal, it is initial in $\DDelta^\op$, hence $B\DDelta^\op \simeq \ast$ is contractible.
\end{proof}

\begin{remark} Removing this injectivity hypothesis is also interesting, since we include the opposite of the map $[1] \to [0]$ --- the category $\DDelta_{\le1}^\op$ is the \textit{split coequalizer} category:
\[\begin{tikzcd}[column sep=small]
    \bullet\rar[bend right=30]\rar[bend left=30] & \bullet\lar
\end{tikzcd} \]
\end{remark}

\begin{proposition} Each of the composites
\begin{align*}
    \DDelta_{\le1}^{\inj,\,\op} \subseteq \DDelta_{\le1}^\op \subseteq \DDelta^\op
\end{align*}
is $1$-cofinal, however none of these inclusions are $\infty$-cofinal.
\end{proposition}
\begin{proof} The universal property of the 1-categorical colimit for both $\DDelta_{\le1}^{\inj,\,\op}$ and $\DDelta_{\le1}^\op$ agree, so this is direct. The other inclusion now follows by \autoref{exa:1-cofinal-parallel-arrows-ddelta}.
\end{proof}

In general we cannot truncate in order to obtain an $\infty$-cofinal diagram. We can, however, restrict only to face maps and throw out degeneracies:

\begin{lemma} \cite[6.5.3.7]{HTT} The inclusion $\DDelta^{\inj,\op} \subseteq \DDelta$ is cofinal.
\end{lemma}



\begin{definition} An $\infty$-category $\mathscr{C}$ is \textit{sifted} if the diagonal map $\mathscr{C} \to \mathscr{C} \times \mathscr{C}$ is cofinal \cite[5.5.8.1]{HTT}.
\end{definition}

\begin{proposition} \cite[5.5.8.11]{HTT} \label{prop:sifted-colimits-finite-limits} 
  Sifted colimits valued in $\Set$ commute with finite products.
\end{proposition}

\begin{remark} In the 1-categorical setting, the converse of \autoref{prop:sifted-colimits-finite-limits} holds, meaning we can take this to be the definition of sifted colimits.
\end{remark}

\begin{example} $\ $
\begin{enumerate}
    \item The category $\DDelta^\op$ is sifted
    \item Any filtered category is sifted
\end{enumerate}
\end{example}

The examples above are the only interesting examples, in the following more precise sense.

\begin{proposition}\label{prop:sifted-is-filtered-plus-geometric-realizations} 
A category $\mathscr{C}$ admits all sifted colimits if and only if it admits all filtered colimits and it admits geometric realizations (meaning $\DDelta^\op$-indexed colimits).
\end{proposition}

\begin{corollary} A 1-category admits all sifted colimits if and only if it admits all filtered colimits and it has coequalizers.
\end{corollary}
\begin{proof} $\DDelta^\op$-indexed colimits are just coequalizers in 1-categories by \autoref{exa:1-cofinal-parallel-arrows-ddelta}.
\end{proof}

Moreover, we may add in \emph{finite coproducts} to capture all colimits.
\def\cC{\mathcal{C}}
\def\cD{\mathcal{D}}
\begin{proposition}\label{prop:colimit-is-sifted-plus-finite-coproducts}
  Let $\cC$ be a category, $\cD \subset \cC$, a full subcategory, and $X$ an object.
  Then, the following are equivalent.
  \begin{enumerate}[label={(\alph*)}]
    \item $X$ is a geometric realization of coproducts of elements in $\cD$.
    \item $X$ is a sifted colimit of finite coproducts of elements in $\cD$.
    \item $X$ is a colimit of object in $\cD$.
  \end{enumerate}
\end{proposition}
\begin{proof}
  The implication (a) $\implies$ (b) is essentially \autoref{prop:sifted-is-filtered-plus-geometric-realizations} (as coproducts are filtered colimits of finite coproducts) and the implication (b) $\implies$ (c) is the fact that colimits of colimits are colimits.
  The implication (c) $\implies$ (a) follows from the Bousfield-Kan formula for a limit of $p\colon K \rightarrow \cC$:
  \[
    \colim_K p \xleftarrow{\;\;\sim\;\;} \colim \left(
\begin{tikzcd}
	{\coprod\limits_{x \in K_0} p(x)} & {\coprod\limits_{\alpha \in K_1} p(\alpha(0))} & {\coprod\limits_{\alpha \in K_2} p(\alpha(0))} & \vdots
	\arrow[shift left, from=1-2, to=1-1]
	\arrow[shift right, from=1-2, to=1-1]
	\arrow[from=1-3, to=1-2]
	\arrow[shift right=2, from=1-3, to=1-2]
	\arrow[shift left=2, from=1-3, to=1-2]
	\arrow[shift left, from=1-4, to=1-3]
	\arrow[shift right, from=1-4, to=1-3]
	\arrow[shift left=3, from=1-4, to=1-3]
	\arrow[shift right=3, from=1-4, to=1-3]
\end{tikzcd}
    \right)
  \] 
  see e.g. \cite[Cor~12.5]{Shah}.
\end{proof}
\subsection{Essential smallness}

Being small is not a property of categories that is invariant under equivalence, so it is more meaningful to ask whether a category is \textit{essentially small} (whether it is equivalent to a small category). This is equivalent to a category admitting a small skeleton, although the axiom of choice is required in order to pick representatives for each isomorphism class of object.

\begin{theorem}\label{thm:finite-type-schemes-essentially-small} 
The category of finite type $S$-schemes is essentially small.
\end{theorem}
\begin{proof}[Proof sketch] (see \href{https://mathoverflow.net/a/251044}{MO251044} for details) If $S = \Spec(A)$ is affine, then the category $\Alg_A^{\text{f.t.}}$ of finite type $A$-algebras is essentially small, since all the objects are isomorphic to algebras of the form $A[x_1, \ldots, x_n]/(f_1, \ldots, f_r)$, of which there are a set. We can glue finite type $S$-schemes over affines, and then bootstrap to the more general case of the base $S$ not being affine by gluing finite type schemes over all affine subschemes of $S$.
\end{proof}

\begin{remark} The category of all $S$-schemes is not essentially small.
\end{remark}


